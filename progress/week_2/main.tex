\documentclass{beamer}

\usepackage{cmap}
\usepackage{mathtext}
\usepackage[T2A]{fontenc}
\usepackage[utf8]{inputenc}
\usepackage[english,russian]{babel}
\usepackage{amsmath,amsfonts,amssymb,amsthm,mathtools}
\usepackage{listings}

\title{Celestial Mechanics: Final Recap}

\usetheme{Berlin}

\begin{document}

\begin{frame}
  \titlepage
\end{frame}

\section{Optimization}

\begin{frame}[fragile]{Problem}
  Сейчас наша модель, в силу своей простоты, довольно медленная и поддерживает симуляцияю лишь
  малого числа частиц (до 1000). Соответсвенно, цель - добваить поодержку для большего числа частиц.
\end{frame}

\begin{frame}[fragile]{Solution}
  Создадим определенный grid в нашем пространстве, разбив его тем самым на блоки. Будем детектить
  столкновения в рамках только одного блока. Сложность все еще останется квадратичной, но константа
  будет сильно меньше.
  \newline
  \newline
  \pause
  Заметим, что этот алгоритм очень хорошо ложится на несколько потоков. Поэтому распараллелим его
  (пока только на CPU).
\end{frame}

\begin{frame}[fragile]{Solution}
  Step 1: Для каждой частицы считаем хеш (который можно воспринимать как просто id блока частицы,
  соответсвенно, у частиц из одного блока хеш будет одинаковый)
  \newline
  \newline
  $HASH = (a.x \div CELLSIZE) \ll XSHIFT \; \vert \; (a.y \div CELLSIZE) \ll YSHIFT \; \vert \; (a.z \div CELLSIZE) \ll ZSHIFT$
  \newline
  \newline
  \pause
  Step 2: Radix Sort по хещам частиц ($O(N)$)
  \newline
  \newline
  \pause
  Step 3: Проверить коллизии для каждого блока
\end{frame}

\begin{frame}[fragile]{Problem}
  Имеет ли смысл параллелить на видеокарте?
  \newline
  \newline
  \pause
  Видимо, нет, ведь, распараллелив алгоритм на CPU, мы уже получили возможность запускать симуляцияю
  на несколько сотен тысяч частиц ($\sim 700000$)
\end{frame}

\section{Joule-Thomson Hypothesis}

\begin{frame}[fragile]{Problem}
  Проверить, выполняется ли гипотеза Джоуля-Томсона (об изменении температуры газа, приближенного по
  свойствам к идеальному, при медленном его просачивании через дырку).
\end{frame}

\begin{frame}[fragile]{Solution}
  Сделаем дырку в нашем сосуде.
  \newline
  \newline
  \pause
  Заметим, что через дырку выходят преимущественно красные частицы (они же горячие, они же быстрые).
  \newline
  \newline
  \pause
  Гипотеза проверена. Profit.
\end{frame}

\end{document}
