\documentclass{beamer}

%%% Работа с русским языком

\usepackage{cmap}					% поиск в PDF
\usepackage{mathtext} 				% русские буквы в формулах
\usepackage[T2A]{fontenc}			% кодировка
\usepackage[utf8]{inputenc}			% кодировка исходного текста
\usepackage[english,russian]{babel}	% локализация и переносы

%%% Дополнительная работа с математикой
\usepackage{amsmath,amsfonts,amssymb,amsthm,mathtools} % пакеты AMS

\usepackage{listings}

\title{Прогресс по модели газа}

\usetheme{Berlin}

\begin{document}
\begin{frame}
  \titlepage
\end{frame}

\section{Что делали}

\begin{frame}[fragile]
  Основа всей модели -- упругий удар двух молекул, для его подсчёта воспользуемся формулой:
  
    \begin{align*}
        v_1' = \frac{2m_2v_2 + v_1(m_1 - m_2)}{m_1 + m_2}\\
        v_2' = \frac{2m_1v_1 + v_2(m_2 - m_1)}{m_1 + m_2}
    \end{align*}
\end{frame}

\begin{frame}[fragile]
  Но, как вы можете заметить, предыдущая формула верна для одномерного случая, но что делать когда скорость имеет большую размерность? (3 в нашем случае)

  Заметим, что в столкновении участвует лишь проекция скорости на ось $\vec{r_2} - \vec{r_1}$, поэтому воспользуемся предыдущей формулой лишь для неё и посчитаем новую скорость:
  \begin{align*}
      v_{proj} = (v,\, \vec{r_2} - \vec{r_1})\\
      v_{proj}' \text{ - получаем из формулы} \Rightarrow\\
      v' = v + v_{proj}' - v_{proj}
  \end{align*}
\end{frame}

\begin{frame}[fragile]
    Теперь обработаем попарно столкновение каждых молекул, а также не забудем про проверку столкновения со стеной. Проверяем независимо по каждой координате:
    \begin{align*}
        \exists i :\: r_i > r_i^{corner} \Rightarrow v_i' = -v_i,\, r_i' = r_i^{corner} 
    \end{align*}
\end{frame}

\begin{frame}[fragile]
    Остаётся собрать метрики!
    \begin{itemize}
        \item<1-> Давление $P = \frac{p}{(\Delta t)\cdot S}$, где $p$ -- импульс стены, полученный за время $\Delta t$.
        \item <2-> Температура $T = \frac{2}{3K}E$, где $E$ -- суммарная кинетическая энергия молекул, а $K$ - всеми известная константа.
        \item <3-> Длина свободного пролёта - между каждой проверкой столкновений засекаем время и считаем среднее расстояние, которое пролетела молекула за время до столкновения.
    \end{itemize}
\end{frame}

\begin{frame}[fragile]
    Что по чиселкам? 

        \begin{itemize}
        \item<1-> Проводим эксперимент в коробке с размерами $(1e^{-8},\,1e^{-8},\,1e^{-8})$
        \item<2-> Тестируем на молекулах гелия - $4$ а.е. с радиус молекул $31e^{-12}$ метра 
        \item<3-> Запускаем пока что 1000 молекул
        \item<4-> Получаем давление $1.9$ МПа, а температуру порядка $\sim 13$ К.
    \end{itemize}
\end{frame}



\section{Какие проблемы}

\begin{frame}
  \begin{itemize}
    \item<1-> Считаем столкновения за квадрат
    \item<2-> Из-за метода подсчёта и слишком большого $\Delta t$ молекулы могут слипаться
  \end{itemize}
\end{frame}


\end{document}